\chapter[Leetcode Algorithms P1-P100]
{Leetcode Algorithms P1-P100
  \label{chLCP1P100}}
\chaptermark{LC P1-P100}



\section{11: Container With Most Water\label{secLCP11CntnrWthMstWtr}}

Given n non-negative integers $a_1$, $a_2$, $\ldots$, $a_n$, where each
represents a point at coordinate $(i,a_i)$. $n$ vertical lines are drawn
such that the two endpoints of line $i$ is at $(i,a_i)$ and $(i,0)$. Find
two lines, which together with $x$-axis forms a container, such that the
container contains the most water.

Note: You may not slant the container and $n$ is at least $2$.

\rrsepline{}

\rrheader{Approach \#1 Brute Force [Time Limit Exceeded]}

\noindent{}\textbf{Algorithm}\\
In this case, we will simply consider the area for every possible pair of
the lines and find out the maximum area out of those.
\begin{lstlisting}[style=raycppnewsnippet]
public class Solution {
  public int maxArea(int[] height) 
  {
    int maxarea = 0;
    for (int i = 0; i < height.length; i++)
      for (int j = i + 1; j < height.length; j++)
        maxarea = Math.max(maxarea, Math.min(height[i],
                                             height[j]) * (j - i));
    
    return maxarea;
  }
}
\end{lstlisting}

\noindent{}\textbf{Complexity Analysis}\\
\begin{itemize}%[noitemsep,topsep=0pt]
\item Time complexity: $\comBigOh{n^2}$. Calculating areas for all
  $\displaystyle \frac{n(n-1)}{2}$ height pairs.
\item Space complexity: $\comBigOh{1}$. Constant extra space used.
\end{itemize}

Note that we only have to consider $\frac{n(n-1)}{2}$ pairs (to calculate
the areas), since once we've calculated the area for $i$ with $j$, the area
for $j$ with $i$ is the same.  That is, they are symmetric.

\rrheader{Approach \#2 (Two Pointer Approach) [Accepted]}

\noindent{}\textbf{Algorithm}\\
The intuition behind this approach is that the area formed between the lines
will always be limited by the height of the shorter line. Further, the
farther the lines, the more will be the area obtained.

We take two pointers, one at the beginning and one at the end of the array
constituting the length of the lines. Further, we maintain a variable
\ctt{maxarea} to store the maximum area obtained till now. At every step, we
find out the area formed between them, update \ctt{maxarea} and move the
pointer pointing to the \textbf{shorter line} towards the other end by one
step.

The algorithm can be better understood by looking at the example below:
\begin{lstlisting}[style=raygeneric]
1 8 6 2 5 4 8 3 7
\end{lstlisting}

See the gif here: https://leetcode.com/problems/container-with-most-water/solution/

How this approach works?

Initially we consider the area constituting the exterior most lines. Now, to
maximize the area, we need to consider the area between the lines of larger
lengths. 

\textbf{If we try to move the pointer at the longer line inwards, we won't
gain any increase in area, since it is limited by the shorter line.}
\rrblue{(Think about it, since moving the longer line will reduce the length
  of the base, but the height is still limited by the shorter line.)}

\textbf{But moving the shorter line's pointer could turn out to be
  beneficial, as per the same argument, despite the reduction in the width.}
\rrblue{(Since we may read a line where the height more than offset the
  reduction in width.)} \emph{This is done since a relatively longer line
  obtained by moving the shorter line's pointer might overcome the reduction
  in area caused by the width reduction.}

For further clarification click
here\footnote{https://discuss.leetcode.com/topic/3462/yet-another-way-to-see-what-happens-in-the-o-n-algorithm}
and for the proof click here\footnote{https://discuss.leetcode.com/topic/503/anyone-who-has-a-o-n-algorithm/2}.

I shall cover both now.

\rrsepline{}

The $\comBigOh{n}$ solution with proof by contradiction doesn't look
intuitive enough to me. Before moving on, read the algorithm first if you
don't know it yet.

Here's another way to see what happens in a matrix representation:

Draw a matrix where the row is the first line, and the column is the second
line. For example, say $n=6$.

In the figures below, $x$ means we don't need to compute the volume for that
case: \textbf{(1)} On the diagonal, the two lines are overlapped;
\textbf{(2)} The lower left triangle area of the matrix is symmetric to the
upper right area.

We start by computing the volume at $(1,6)$, denoted by $o$. Now, if the
left line is shorter than the right line, then all the elements left to
$(1,6)$ on the first row have smaller volume \rrblue{(because in those, we
  keep the shorter one still $(i=1)$ but we move the longer one $(j=6)$,
  thus reducing the width)}, so we don't need to compute those cases
(crossed by $-{}-{}-$).
\begin{lstlisting}[style=raygeneric]
  1 2 3 4 5 6
1 x ------- o
2 x x
3 x x x 
4 x x x x
5 x x x x x
6 x x x x x x
\end{lstlisting}
Next we move the left line and compute $(2,6)$. Now if the right line is
shorter, all cases below $(2,6)$ are eliminated. \rrblue{(If the right line
  is shorter, then the same principal applies, there's no point moving the
  left line $i=2$ towards $j=6$, because we just reduce the width without
  being able to increase the max length of the two lines)}
\begin{lstlisting}[style=raygeneric]
  1 2 3 4 5 6
1 x ------- o
2 x x       o
3 x x x     |
4 x x x x   |
5 x x x x x |
6 x x x x x x
\end{lstlisting}
And no matter how this $o$ path goes, we end up only need to find the max
value on this path, which contains $n-1$ cases.
\begin{lstlisting}[style=raygeneric]
  1 2 3 4 5 6
1 x ------- o
2 x x - o o o
3 x x x o | |
4 x x x x | |
5 x x x x x |
6 x x x x x x
\end{lstlisting}
Hope this helps. I feel more comfortable seeing things this way.

Let's code this up.





















































